%------------------------------------------------
\begin{frame}{iTunes not meant for missiles}
  \small
  \begin{quote}
    "You also agree that you will not use these products for any purposes prohibited by United States law, including, without limitation, the development, design, manufacture or production of \textbf{missiles}, or \textbf{nuclear, chemical or biological weapons}."
  \end{quote}
  \mbox{}\hfill\textit{--- Apple iTunes EULA, 2007} \cite{ITunes}
\end{frame}


%------------------------------------------------
\begin{frame}{Practical Interpretation}
  \begin{itemize}
    \item Reaffirms U.S.\ export--control rules: U.S.\ technology may not be employed in weapons of mass destruction programmes or their delivery systems.
    \item Places the responsibility on the end\textemdash{}user to ensure compliance with the \emph{Export Administration Regulations} and related legislation.
    \item Signals Apple's intent to distance ordinary consumer software from potential dual\textemdash{}use or military applications.
  \end{itemize}
\end{frame}

%------------------------------------------------
\begin{frame}{Why Is It There?}
  \begin{itemize}
    \item \textbf{Regulatory Compliance} \textemdash{} blanket language satisfies U.S.\ export\textendash{}control requirements.
    \item \textbf{Risk Mitigation} \textemdash{} reduces liability should the software be diverted to a sanctioned programme.
    \item \textbf{Industry Standard} \textemdash{} similar clauses appear in Adobe, Microsoft and Google EULAs to streamline export due\textemdash{}diligence.
  \end{itemize}
\end{frame}

\begin{frame}{Disney+ clause}
  \small
  \begin{quote}
    ANY DISPUTES BETWEEN YOU AND US, EXCEPT DISPUTES RESOLVED IN SMALL CLAIMS COURT OR RELATING TO THE OWNERSHIP OR ENFORCEMENT OF INTELLECTUAL PROPERTY RIGHTS, ARE SUBJECT TO A CLASS ACTION WAIVER AND MUST BE RESOLVED BY INDIVIDUAL BINDING ARBITRATION.
  \end{quote}
  \mbox{}\hfill\textit{--- Disney Terms of Use, §8, May~24~2024}
\end{frame}

%------------------------------------------------
\begin{frame}{Practical Interpretation}
  \begin{itemize}
    \item \textbf{Universal reach} \textemdash{} Applies to \emph{all} Disney entities and disputes, including personal injury and wrongful death.
    \item \textbf{Private forum} \textemdash{} Jury and class\textendash{}action rights are waived; claims proceed in individual arbitration.
    \item \textbf{Opt\textendash{}out window} \textemdash{} Users have 30 days to opt out, but few are aware of the provision.
  \end{itemize}
\end{frame}

%------------------------------------------------
\begin{frame}{Case Study: incident at disneyland}
  \small
  \begin{itemize}
    \item Widower Jeffrey Piccolo sued after his wife died from an allergic reaction at Disney Springs (Oct~2023).
    \item Disney’s August~2024 motion sought dismissal or arbitration, citing Piccolo’s 2019 Disney+ sign\textendash{}up.
    \item Following public backlash and legal scrutiny, Disney withdrew the motion a week later. \cite{Disney}
  \end{itemize}
\end{frame}

%------------------------------------------------
\begin{frame}{Why Is It There?}
  \begin{itemize}
    \item \textbf{Cost Control} \textemdash{} Arbitration is faster and less expensive than jury trials.
    \item \textbf{Risk Containment} \textemdash{} Eliminates class actions and caps unpredictable punitive damages.
    \item \textbf{Industry Standard} \textemdash{} Netflix, Amazon, and other platforms use similar clauses to streamline dispute resolution.
  \end{itemize}
\end{frame}
