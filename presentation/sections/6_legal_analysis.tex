%------------------------------------------------
\begin{frame}{Are "Quirky" Clauses Legally Binding?}
  \begin{itemize}
    \item \textbf{Adhesion Contracts} – EULAs are take‑it‑or‑leave‑it; enforceability hinges on notice and assent.
    \item \textbf{Surprising Terms Rule} – Unusual or onerous terms must be conspicuous; otherwise they risk invalidation (see \href{https://en.wikipedia.org/wiki/Tilden_Rent-A-Car_Co_v_Clendenning}{\textit{Tilden Rent‑A‑Car v. Clendenning} (1978)} and \href{https://law.justia.com/cases/federal/appellate-courts/F3/306/17/59077/}{\textit{Specht v. Netscape} (2002)}).
    \item \textbf{Unconscionability / Unfair Terms} – U.S. courts apply the “shock‑the‑conscience” test; the EU relies on \href{https://eur-lex.europa.eu/legal-content/EN/TXT/?uri=CELEX:31993L0013}{Directive 93/13/EEC on Unfair Terms in Consumer Contracts}.
    \item \textbf{Public‑Policy Limits} – Clauses that waive fundamental rights (e.g., personal‑injury liability) or contravene mandatory law are void.
  \end{itemize}
\end{frame}

%------------------------------------------------
\begin{frame}{What Courts and Lawyers Say}
  \small
  \begin{itemize}
    \item \href{https://law.justia.com/cases/federal/appellate-courts/F3/86/1447/512628/}{\textit{ProCD v. Zeidenberg} (7th Cir. 1996)} – Shrink‑wrap license upheld where consumer could return the product.
    \item \href{https://law.justia.com/cases/federal/appellate-courts/F3/763/1171/178520/}{\textit{Nguyen v. Barnes \& Noble} (9th Cir. 2014)} – Browse‑wrap failed: a passive link was insufficient for assent.
    \item \href{https://eur-lex.europa.eu/legal-content/EN/TXT/?uri=CELEX:62015CJ0191}{\textit{VKI v. Amazon EU} (CJEU 2016)} – Foreign‑law clauses may be unfair if not individually negotiated.
    \item Scholars highlight a “\textit{duty to draft clearly}” (see \href{https://global.oup.com/academic/product/wrap-contracts-9780199336757}{Kim 2018}) and advocate plain‑language summaries such as the \href{https://www.engadget.com/2022-01-10-tldr-act-terms-of-service-plain-language.html}{TL;DR Act 2022} proposal.
  \end{itemize}
\end{frame}

%------------------------------------------------
\begin{frame}{Ethics \& Power Asymmetry}
  \begin{itemize}
    \item \textbf{Information Imbalance} – Empirical study finds fewer than 0.2 \% of shoppers open EULAs (\href{https://papers.ssrn.com/sol3/papers.cfm?abstract_id=160308}{Bakos et al., 2014}).
    \item \textbf{No Real Negotiation} – Consumers lack bargaining power; firms have inserted “immortal soul” or “zombie apocalypse” clauses without challenge.
    \item \textbf{Autonomy vs. Paternalism} – Is “click \textsc{I Agree}” genuine consent when crucial stakes are obscured?
    \item \textbf{Fairness Heuristics} – Transparency, salience and meaningful opt‑outs can help restore contractual balance (see \href{https://www.loeb.com/en/insights/publications/2022/01/tldr-act}{Loeb \& Loeb 2022} for design suggestions).
  \end{itemize}
\end{frame}

%------------------------------------------------
\begin{frame}{Key Takeaways}
  \begin{enumerate}
    \item Novel clauses survive only with clear notice and if not unconscionable.
    \item Courts increasingly scrutinise browse‑wrap and surprise terms; UX designers must build explicit assent flows.
    \item Ethical drafting demands plain language, proportionality, and respect for consumer autonomy.
  \end{enumerate}
\end{frame}
